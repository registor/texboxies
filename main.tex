% 使用ctexart文档类(用XeLaTeX编译,直接支持中文)
\documentclass{ctexart}

%导言区,可以在此引入必要的宏包
\usepackage{texboxie}

\usepackage{hyperref}

% ========设置标题的格式========
\ctexset{
  section = {
    format+ = \zihao{-4} \heiti \raggedright,
    name = {,、},
    number = \chinese{section},
    beforeskip = 1.0ex plus 0.2ex minus .2ex,
    afterskip = 1.0ex plus 0.2ex minus .2ex,
    aftername = \hspace{0pt}
  },
  subsection = {
    format+ = \zihao{5} \heiti \raggedright,
    % name={\thesubsection、},
    name = {,、},
    number = \arabic{subsection},
    beforeskip = 1.0ex plus 0.2ex minus .2ex,
    afterskip = 1.0ex plus 0.2ex minus .2ex,
    aftername = \hspace{0pt}
  }
}

\newcommand{\qtmark}[1]{``#1''}

\title{\Large \heiti 采用tcolorbox宏包设计的用于同时排版\LaTeX 代码片段及其排版结果的宏包texboxie.sty}
\author{\zihao{4} \fangsong 耿楠\\\small \songti 西北农林科技大学信息
  工程学院,陕西$\cdot$杨凌,712100}

\begin{document} %在document环境中撰写文档

\maketitle

\begin{abstract}
  在阅读其暨南大学数学系吕荐瑞的\LaTeX 文档排版教程及其源代码
  (\url{https://lvjr.bitbucket.io/latex.html?2018})时,发现用
  \qtmark{盒子}排版\LaTeX 代码片段及其排版结果的方式在撰写\LaTeX 相关文档文档时非
  常有效。在查看其源代码时,发现仅使用了\qtmark{listings}宏包实现了
  \LaTeX 代码的排版,而未使用更为方便的\qtmark{minted}宏包,为此,结合
  本人曾开发的boxiesty宏包(\url{https://github.com/registor/boxiesty}),设计了该texboxie宏包。

  该宏包可以为经常需要编写\LaTeX 代码的排版人员提供帮助,但由于作
  者水平有限,一定有不足之处,欢迎大家多提宝贵意见和建议。
\end{abstract}

\section{使用样例}

该宏包主要定义了\env{codeonly}、\env{outonly}、\env{texdemoh}和
\env{texdemoh}四个环境分别输出不同形式的代码片段及其排版结果。

如果编译\LaTeX 源代码时指定了
\qtmark{--shell-escape}选项,则使用\qtmark{minted}宏包排版\LaTeX 源代
码,否则,则使用\qtmark{listings}宏包排版\LaTeX 源代码。

\subsection{\env{codeonly}环境}

该环境仅输出\LaTeX 源代码,其基本语法为:

\begin{command}
\cmd{begin}\marg{codeonly} \\
\ldots \\
\cmd{end}\marg{codeonly} 
\end{command}

该环境无需指定参数,其排版效果如:

\begin{codeonly}
如果$p$是素数,$\gcd(a,p)=1$,则有
$$a^{p-1} \equiv 1 \pmod{p}$$
\end{codeonly}

\subsection{\env{outonly}环境}

该环境仅输出\LaTeX 源代码片段的排版结果,其基本语法为:

\begin{command}
\cmd{begin}\marg{outonly} \\
\ldots \\
\cmd{end}\marg{outonly} 
\end{command}

该环境无需指定参数,其排版效果如:

\begin{outonly}
如果$p$是素数,$\gcd(a,p)=1$,则有
$$a^{p-1} \equiv 1 \pmod{p}$$
\end{outonly}

\subsection{\env{texdemoh}环境}

该环境在水平方向同时输出\LaTeX 源代码片段及其排版结果,其基本语法为:

\begin{command}
\cmd{begin}\marg{texdemoh}\oarg{options1}\oarg{options2} \\
\ldots \\
\cmd{end}\marg{texdemoh} 
\end{command}

其中,\Arg{options1}确定要不要分割虚线,省略该选项则表示需要分割虚线,
使用\qtmark{*}表示不需要分割虚线,\Arg{options2}为左边占有的宽度比例,
默认为0.6,其排版效果如:

\begin{texdemoh}[0.7]
如果$p$是素数,$\gcd(a,p)=1$,则有
$$a^{p-1} \equiv 1 \pmod{p}$$
\end{texdemoh}

\begin{texdemoh}*[0.55]
如果$p$是素数,$\gcd(a,p)=1$,则有
$$a^{p-1} \equiv 1 \pmod{p}$$
\end{texdemoh}

\subsection{\env{texdemov}环境}

该环境在垂直方向同时输出\LaTeX 源代码片段及其排版结果,其基本语法为:

\begin{command}
\cmd{begin}\marg{texdemoh}\oarg{options} \\
\ldots \\
\cmd{end}\marg{texdemoh} 
\end{command}

其中,\Arg{options}确定要不要分割虚线,省略该选项则表示需要分割虚线,
使用\qtmark{*}表示不需要分割虚线,其排版效果如:

\begin{texdemov}
  如果$p$是素数,$\gcd(a,p)=1$,则有
$$a^{p-1} \equiv 1 \pmod{p}$$
\end{texdemov}

\begin{texdemov}*
如果$p$是素数,$\gcd(a,p)=1$,则有
$$a^{p-1} \equiv 1 \pmod{p}$$
\end{texdemov}

\section{说明}
本宏包建议使用\qtmark{minted}宏包实现代码的排版,用\qtmark{xelatex
  --shell-escape main.tex}编译tex文件,但minted
需要的python及其pygments模块,请提前安装该模块。

若在编译是不使用\qtmark{--shell-escape}参数,则会自动切换到用listings
排版代码,注意有部分代码名称与pygments定义不一致,请自行查阅相关手册。

\end{document}

%%% Local Variables:
%%% mode: latex
%%% TeX-master: t
%%% End:
